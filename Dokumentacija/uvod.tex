\hspace{\parindent} Metaheuristika se odnosi na oblikovni obrazac ugrađivanja prethodno stečenog znanja u izgradnju optimizacijskih algoritama. Smatra se potpodručjem stohastičke optimizacije. Stohastička optimizacija je kombinacija algoritama i slučajnih procesa radi pronalaženja optimalnih rješenja. 

Prirodom nadahnute metaheuristike su one koje crpe ideje za pretraživanje prostora stanja iz prirode. Algoritmi rojeva su potkategorija prirodom nadahnutih metaheuristika. Svoja svojstva crpe iz kolektivnog ponašanja životinja i kukaca. Roj se sastoji od više individua čije je ponašanje vrlo jednostavno, ali zajedno mogu djelovati inteligentno. Postoje mnogi takvi algoritmi, najpoznatiji su mravlji algoritam i algoritam pčela. 

U ovom seminaru razmatraju se manje poznati algoritmi: algoritam šišmiša i algoritam kukavičjeg pretraživanja. Oba su algoritma opisana i prikazan je njihov pseudokod. Navedeni su načini na koje se algoritmi mogu lokalno pokrenuti i dani su primjeri raznih primjena algoritama. 