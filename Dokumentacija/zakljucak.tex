

\hspace{0.5cm}Oba algoritma (algoritam šišmiša i algoritam kukavičjeg pretraživanja), unatoč vrlo jednostavnim formulacijama, pokazuju obećavajuće rezultate na raznim problemima od znanstvenih i matematičkih do industrijskih i komercijalnih.

\hspace{0.5cm} Prednost algoritma šišmiša je mogućnost ugađanja omjera između istraživačkog i eksploatacijskog ponašanja šišmiša putem parametara eholokacije. Nadalje, jednostavno ga je ostvariti i nije računalno zahtjevan. Algoritam je već poznat te se može pronaći u različitim knjižnicama optimizacijskih algoritama, poput MEALPY, što olakšava primjenu. Pokazuje se korisnim u primjeni na različite optimizacijske probleme. 


\hspace{0.5cm} Algoritam kukavičjeg pretraživanja, kao i algoritam šišmiša, nije računalno zahtjevan i jednostavno ga je ostvariti. Može se pronaći u brojnim knjižnicama optimizacijskih algoritama, uključujući i MEALPY. Može se primijeniti na optimizacijske probleme u različitim domenama s dobrim rezultatima. 



\hspace{0.5cm}Općenito, prirodnom nadahnuti metaheuristički algoritmi rojeva na učinkovit način pronalaze neko rješenje problema. Takvo rješenje neće nužno biti najbolje moguće, ali može biti dovoljno dobro za pojedino područje primjene. Daljnjim istraživanjem mogu se pronaći učinkovitiji algoritmi i poboljšati postojeći.